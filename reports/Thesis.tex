\documentclass[]{article}

\usepackage[a4paper, margin=1in]{geometry}

\usepackage{graphicx}
\usepackage{caption}
\usepackage{subcaption}
\usepackage{setspace}
\usepackage{hyperref}
\usepackage{outlines}
\usepackage{longtable}
\usepackage{pdflscape}
\usepackage{listings}

\usepackage[final]{pdfpages}

\newcommand{\mnras}{Monthly Notices of the Royal Astronomical Society}
\newcommand{\aap}{Astronomy and Astrophysics}
\newcommand{\aj}{The Astronomical Journal}
\newcommand{\actaa}{Acta Astronomica}
\newcommand{\apj}{Astrophysical Journal}
\newcommand{\rmxaa}{Revista Mexicana de Astronom\'ia y Astrof\'isica}

\pagenumbering{roman}

%opening
\title{}
\author{Christopher Wells}

\begin{document}

\begin{center}
	\vspace*{200pt}
	
	\textbf{Clustering Fundamental-mode RR Lyrae in the Magellanic Clouds to Determine a New Boundary Line for the Oosterhoff Dichotomy}
	
	\vspace{20pt}
	
	Christopher Wells
	
	Candidate for B.A. Degree
	
	in Computer Science
	
	\vspace{20pt}
	
	State University of New York, College at Oswego
	
	College Honors Program
	
	\vspace{20pt}
	
	December, 2018
\end{center}

\newpage

\singlespacing

\begin{abstract}
	
	RR Lyrae are a type of periodic variable star that are often examined due to their utility as distance indicators in space. An effect known as the Oosterhoff Dichotomy has been seen in populations of RR Lyrae stars in globular clusters in the Milky Way Galaxy. This phenomenon is thought to have some importance in learning more about the formation of the Milky Way Galaxy.

\end{abstract}

\newpage

\doublespacing

\tableofcontents

\newpage

\pagenumbering{arabic}

\section{Advice to Future Honors Thesis Students}

\newpage

\section{Acknowledgements}

\newpage

\section{Author's Reflections}

\newpage

\section{RR Lyrae}

RR Lyrae are a type of radially pulsating, periodic variable star that are numerous in Local Group galaxies and examined due to their utility as distance indicators and indicators of early stellar composition. \cite{szczygiel_2009, kinemuchi_2006_a} (Szczygiel et al. 2009, Kinemuchi et al. 2006 (Analysis of ...)) RR Lyrae are advanced age stars (> 10 Gyrs) that undergo radial pulsations, whereby they expand and contract radially over time. (Szczygiel et al. 2009, ?) These pulsations that RR Lyrae undergo cause the brightness of the star to change over time, with it becoming brighter as it expands and dimmer as it contracts. (?) This change in brightness over time classifies RR Lyrae as variable stars, however unlike some variable stars, RR Lyrae change in brightness at a regular interval, or period, which makes them periodic variable stars. (?) RR Lyrae are numerous in the Milky Way and other Local Group galaxies and are additionally easy to identify. (Soszyński, 2016, Kinemuchi et al. 2006 (Analysis of ...)) RR Lyrae are often researched due to their use as “standard candles”, or distance indicators.\cite{soszynski_2016} (Soszyński, 2016) The intensity of the luminosity change caused by the periodic pulsations is correlated with the period of the pulsations, thus the period can be used to calculate the luminosity of the star which can be compared with the observed brightness of the star to determine its distance. (?)

\subsection{Light Curves}

When examining variable stars, such as RR Lyrae, we often are working with data on the brightness of the variable star over a given period of time. Sky surveys that gather data on variable stars take photometric brightness measurements of many different variable stars at various different points in time. These brightness measurements over time for a particular variable star are called a light curve. (?)

By examining the light curve of a variable star we can get an idea of the type of brightness variation that the variable star is undergoing. Different types of variable stars tend to have different characteristic patterns of brightness variation. (?)

Light curves are typically plotted as a scatter plot where the x-axis is the time of the observation in Mean Julian Days (MJD) and the y-axis is the brightness of the star in magnitude. (?)

In the case of periodic variable stars, like RR Lyrae, a transformation called period folding is often applied to the light curve to look specifically at the periodic variation that the star undergoes. To period fold a light curve, the time for each brightness measurement is “folded” over the period of the star by taking the remainder of the division of the time by the period to get the “phase” of the observation. \footnote{For non-periodic variable stars, period folding of the light curve is not possible, since the star does not have a variation period to fold the observations over. (?)} (?)

This “period folded light curve” makes it easier to see the pattern of variation that a periodic variable star undergoes, since some periodic variable stars undergo variations that have a higher frequency than the frequency at which the survey observes them. For such periodic variable stars it is difficult to see the variation pattern as each instance of the pattern may have only a few observations. However, when the light curve is period folded, the different instances of the variation pattern line up over one another, making the repeated pattern more apparent. So the low sampling of each instance of the pattern is countered by the high sampling of all of the instances of the pattern as a whole.

\begin{figure}
	\centering
	\begin{subfigure}{.5\textwidth}
		\centering
		\includegraphics[width=6cm]{figures/light_curve_examples/light_curve_raw_i.png}
		\label{fig:light_curve_raw_i}
	\end{subfigure}%
	\begin{subfigure}{.5\textwidth}
		\centering
		\includegraphics[width=6cm]{figures/light_curve_examples/light_curve_folded_i.png}
		\label{fig:sub2}
	\end{subfigure}
	\caption{Plots of the light curve (left) and period folded light curve (right) for the RRab (OGLE-LMC-RRLYR-00001) from the OGLE IV LMC dataset. In the light curve (left) the variance in the RRab’s brightness over time is visible, however, it is only in the period folded light curve (right) that the pattern of the variability is clearly shown.}
	\label{fig:light_curves}
\end{figure}

\subsection{Properties \& Sub-Types}

RR Lyrae are characterized mainly through a few different properties, both physical and those observed through their light curves. (?) One of the most commonly examined properties of RR Lyrae is their period, that is the time interval at which they repeat their pulsations and variance in brightness. (?) RR Lyrae typically have periods of 0.2 to 1.2 days. (Szczygiel et al. 2009) Another commonly examined property is amplitude, or the amount of change that the brightness of the star undergoes as a part of its regular pattern. (?) RR Lyrae typically have amplitudes of 0.2 to 1.6 mag in V photometric band light. (Szczygiel et al. 2009)

… … … (metalicity, etc… ?)

[plot of RR Lyrae light curve with period and amplitude annotated]

There are several different sub-types of RR Lyrae that tend to have different characteristics due to the mode(s) that they pulsate in. (Chen, 2013) RR Lyrae that pulsate in the fundamental mode are known as RRab. RR Lyrae that pulsate in the first overtone mode are known as as RRc. RR Lyrae that pulsate in both the fundamental and first overtone modes are known as RRd. \footnote{This letter-based naming convention came from ...} \footnote{An additional sub-type of RR Lyrae called RRe, which had second overtone mode pulsations, was thought to exist. However, the existence of RRe has been debated in the literature. (Chen, 2013) For the OGLE IV data release, OGLE has stopped classifying stars as RRe, choosing to group in all stars previously classified as RRe to be RRc, so as to not separate the first and second overtone pulsators. They noted that they did this “because of the doubts whether the RRe stars exist at all”. (Soszyński, 2016)} \footnote{An alternate naming scheme using numbers is sometimes used. In this scheme RRab are called RR0, RRc are called RR1, RRd are called RR01, and RRe are called RR2. This system was introduced in Alcock et al. (2000) in order to allow for easier mnemonic memorization of the different types, with the 0 indicating fundamental mode, 1 indicating first overtone mode, and 2 indicating second overtone mode. (Chen, 2013) I have decided to use the letter-based naming convention instead due to OGLE’s use of it and since it seem to be the more popular naming scheme in the literature.} (Chen, 2013)

These different type of RR Lyrae have different ranges of values for properties like period and amplitude. …. …. ….


\begin{center}
\begin{tabular}{|l|c|c|c|c|}
	\hline
	Sub-type & Mean FM Period & Mean FM I-band Amplitude & Mean FO Period & Mean FO I-band Amplitude \\
	\hline
	RRab & 0.58 & 0.53 & - & - \\
	RRc & - & - & 0.33 & 0.26 \\
	RRd & 0.49 & 0.16 & 0.36 & 0.26 \\
	\hline
\end{tabular}
\captionof{table}{Summary information on the properties of the RR Lyrae sub-types calculated using the data from the OGLE IV data release.\cite{soszynski_2016} Column names containing “FM” refer to the fundamental mode pulsations and column names containing “FO” refer to the first overtone mode pulsations. A dash indicates that a column is not applicable for a particular sub-type.}
\end{center}

\newpage

\section{Oosterhoff Dichotomy}

In 1939, Pieter Oosterhoff published a paper on some observations he made about RR Lyrae in various globular clusters in the Milky Way galaxy. Oosterhoff noticed that if you examined the mean periods of the RRab and RRc in globular clusters, the globular clusters appeared to cluster into two groups. He found that M5 and M3 had lower RRab and RRc periods, while M53, M15, and ω Cen had higher RRab and RRc periods. \footnote{Globular clusters are typically referred to either by their NGC identifier (like NGC 5024), an M followed by a number (like M53), or a specific name (like ω Cen). (?) See the appendix ... … … … for a list of globular clusters and their various names.} (Oosterhoff, 1939) These two groups later became known as the Oostherhoff I (OoI) and Oosterhoff II (OoII) groups. (?)

\begin{figure}
	\centering
	\includegraphics[width=8cm]{figures/globular_clusters/oosterhoff_1939.png}
	\caption{A plot of the Milky Way globular clusters examined in Oosterhoff (1939). A gap is visible between the clusters of M5 + M3 and M53 + M15 + $\omega$ Cen.}
	\label{fig:oosterhoff_1939_globular_clusters}
\end{figure}

While Oosterhoff originally examined globular clusters in terms of the mean and median RRab and RRc periods, research since has looked at other parameters as well in terms of the Oosterhoff dichotomy. (?) ...

However, examinations of globular clusters outside of the Milky Way found some globular clusters that exist in between these two groups, in the Oosterhoff gap. (?) Globular clusters that fell in between the Oosterhoff I and Oosterhoff II groups became known as Oosterhoff intermediate (Oo-Int) globular clusters. (?)

\begin{figure}
	\centering
	\includegraphics[width=17cm]{figures/globular_clusters/globular_clusters_by_oosterhoff_type.png}
	\caption{A plot of modern data on globular clusters showing the Oosterhoff classifications for each that are agreed upon in the literature. Shows both Milky Way and non-Milky Way globular clusters. ``Conflicted'' indicates that multiple different classifications have been suggested in different papers. (see section ~\ref{sec:app_globular_clusters_summary}~ \nameref{sec:app_globular_clusters_summary})}
	\label{fig:modern_globular_clusters}
\end{figure}

\begin{figure}
	\centering
	\includegraphics[width=17cm]{figures/globular_clusters/globular_clusters_by_location.png}
	\caption{A plot of the modern data on globular clusters with agreed upon Oosterhoff classifications, broken up into Milky Way and non-Milky Way globular clusters. Shows that the plotted Oosterhoff Intermediate clusters are all not in the Milky Way galaxy. (see section ~\ref{sec:app_globular_clusters_summary}~ \nameref{sec:app_globular_clusters_summary})}
	\label{fig:modern_globular_clusters_location}
\end{figure}

\newpage

\section{Clustering}

Clustering is a process of applying algorithms to data in order to separate a dataset into multiple sub groups, or clusters, that have differing properties. (?) The idea is to partition the data into clusters such that in cluster distance is minimized, or rather the in group similarity should be high. Additionally the intracluster distance should be maximized, or rather the similarity between things in different clusters should be low. (?)

By applying clustering algorithms to a dataset, we can learn more about sub-categories that exist in the data. (?) … … …

The clustering algorithm that I used for the data analysis I performed was K-Means clustering. K-Means clustering is an algorithm that takes a number of clusters to look for (called k) and works by making random initial guesses for the k clusters and gradually refining them until a convergence threshold or iteration limit is reached. (?) … (more on K-Means algorithm)


… … …


\newpage

\section{Data Analysis}

To learn more about the Oosterhoff dichotomy I wanted to try applying clustering algorithms to the fairly recent OGLE IV dataset to see if I could separate the RRab into two clusters that have properties that correspond to the two main Oosterhoff groups.

\subsection{OGLE IV Dataset}

OGLE IV is a dataset that includes data on RR Lyrae present in the Large and Small Magellanic Clouds. (?) The Large and Small Magellanic clouds, also referred to as the LMC and SMC respectively, are two galaxies in the Local Group that orbit the Milky Way galaxy. (?) They are of particular interest because of their proximity to the Milky Way. (?)

The OGLE IV dataset contains data on a total of 45,451 RR Lyrae, 39,082 in the LMC and 6,369 in the SMC. \cite{soszynski_2016} … (add breakdown by cluster and RR type)

The OGLE IV dataset contains light curves for the RR Lyrae in both I and V photometric bands. The I band light curves are highly sampled over time, while the V band light curves are more sparse. \cite{soszynski_2016}

\begin{center}
	\begin{tabular}{|l|l|l|l|}
		\hline
		\multicolumn{4}{|c|}{LMC} \\
		\hline
		Band & Num RR Lyrae & Mean \# Observations & Std Dev Observations \\
		\hline
		I & 39,609 & 491 & 216 \\
		V & 37,169 & 100 & 79 \\
		\hline
		\multicolumn{4}{|c|}{SMC} \\
		\hline
		Band & Num RR Lyrae & Mean \# Observations & Std Dev Observations \\
		\hline
		I & 6,560 & 404 & 150 \\
		V & 6,308 & 37 & 32 \\		
		\hline
	\end{tabular}
	\captionof{table}{OGLE IV Light Curve Observation Count Summaries}
\end{center}

\subsection{Applying Clustering}
When I was analyzing the data from the OGLE IV dataset I focused on the RRab in particular. This was in part due to the dataset having a larger number of RRab light curves than RRc or RRd. (?)

In working with the OGLE IV data, there were a few parameters of RRab that I was looking at in particular.

…
…
...

I wanted to find a way to separate the RRab population in a way that transfers beyond a particular one of Magellanic clouds

…
…
…

Initially I looked into a few different clustering techniques to see if any of them would be good to try on the data. I tried using K-Means clustering and Spectral Clustering, but neither of these techniques achieved the separation I was looking for.

Looking more closely at the data, I noticed that the groups seemed to follow a trend in period and I band amplitude. Additionally the two groups seemed to be separated by a boundary that follows a similar shape to the trend in the data. So I figured that it would be best to try using a clustering technique that would fit clusters based on a boundary with such a shape.

Many clustering techniques like K-Means clustering tend to fit “blob-like” clusters, so I figured that using such a technique on its own would not be the best approach due to the apparent shape of the groups having a “non-blob” shape.

So I decided to try taking an approach similar to how a kernel function is used, where I would pre-transform the data points into a feature space where clustering algorithms would be able to more easily fit a boundary line of a particular shape.

In order to determine the transformation to apply to the data before using the clustering algorithm, I fit a polynomial trend line to the data. In order to do this, I fit the trendline to only the more densely populated regions, as the more sparse regions caused odd behavior in the trend line.

… … …

With the trend line, I could subtract out the influence line from the data to get a 1 dimensional space, where a boundary value would separate out the original data in a line with the shape of the fitted trend line. Then I was able to apply K-Means clustering to the data to achieve the following clusters.

[add image]

… … …


\newpage

\singlespacing

\section{References}

\begingroup
\renewcommand{\section}[2]{}
\bibliographystyle{plain}
\bibliography{references}
\endgroup

\newpage

\section{Appendices}

\subsection{Collected Globular Cluster Information}

\subsubsection{Globular Clusters Summary}
\label{sec:app_globular_clusters_summary}

This table contains information on various globular clusters, including information such as their location, suggested Oosterhoff classifications, and various summary statistics on their RRab and RRc.

\vspace{12pt}

Each row in the table contains information on one particular globular cluster.

\vspace{12pt}

This data was collected by examining different papers related to the Oosterhoff dichotomy and collecting data that was listed in various tables about globular clusters. This information was then put into a spreadsheet and filtered down to globular clusters that met the following inclusion criteria.

\begin{itemize}
	\item I was able to find at least one suggested Oosterhoff classification
	\item The Oosterhoff classification was not ``mixed''
	\item I was able to find information about the location of the globular cluster
\end{itemize}

\textbf{Link:} \url{https://github.com/ExcaliburZero/rr-lyrae-oosterhoff-clustering/blob/master/data/raw/gc_oosterhoff/Collected\%20Globular\%20Cluster\%20Information\%20-\%20Globular\%20Clusters\%20Summary.csv}

\paragraph{Columns}

\begin{outline}
	\1 Name
	\2 The non-NGC name of the globular cluster. (if any) (ex. M3, M5, M72)
	\1 NGC
	\2 The NGC designation of the globular cluster, or the globular cluster’s name if an NGC designation could not be found. (ex. NGC 5272, NGC 5904, NGC 6121)
	\1 GC Location
	\2 The location of the globular cluster. (ex. Milky Way, LMC, Fornax dSph, ...)
	\1 GC Location Source
	\2 The paper that mentions the location of the globular cluster.
	\1 Num Oost Types
	\2 The number of different Oosterhoff classifications that were mentioned by papers for the globular cluster.
	\1 Num Oost Sources
	\2 The number of different papers that mentioned an Oosterhoff classification for the globular cluster.
	\1 Oost Type
	\2 The Oosterhoff classification mentioned in papers for the globular cluster. (ex. I, II, III, Oo-Int, Conflicted)
	\2 ``Conflicted'' indicates that more than one Oosterhoff classification has been suggested for the globular cluster.
	\1 Mean RRab Period
	\2 The mean period of known RRab stars in the globular cluster.
	\1 Mean RRab Period Source
	\2 The paper that mentions the ``Mean RRab Period'' of the globular cluster.
	\1 Mean RRc Period
	\2 The mean period of known RRc stars in the globular cluster.
	\1 Mean RRc Period Source
	\2 The paper that mentions the ``Mean RRc Period'' of the globular cluster.
	\1 Metalicity
	\2 The iron abundance/metallicity of the globular cluster.
	\1 Metalicity Source
	\2 The paper the mentions the ``Metallicity'' of the globular cluster.
	\1 \# RRab
	\2 The number of known RRab stars in the globular cluster.
	\1 \# RRab Source
	\2 The paper that mentions the ``\# RRab''.
\end{outline}

\newpage

\subsubsection{Oosterhoff Classifications}
\label{sec:app_oosterhoff_classifications}

This table contains information on mentions of the Oosterhoff classifications of various globular clusters. This information was used to create the ``Num Oost Types'', ``Num Oost Sources'', and ``Oost Type'' columns in the ``Globular Clusters Summary'' table.

\vspace{12pt}

Additionally this table shows that there are some globular clusters that have had multiple different Oosterhoff classifications suggested by different papers in the literature.

\vspace{12pt}

This data was collected by looking through papers related to the Oosterhoff dichotomy and looking for mentions of the Oosterhoff classifications of globular clusters in the text of the paper or in tables in the paper. This information was then put into a spreadsheet and filtered down to globular clusters that met the same inclusion criteria as the ``Globular Clusters Summary'' table.

\vspace{12pt}

\textbf{Link:} \url{https://github.com/ExcaliburZero/rr-lyrae-oosterhoff-clustering/blob/master/data/raw/gc_oosterhoff/Collected\%20Globular\%20Cluster\%20Information\%20-\%20Oosterhoff\%20Classifications.csv}

\paragraph{Columns}

\begin{outline}
	\1 Name
	\2 The non-NGC name of the globular cluster. (if any) (ex. M3, M5, M72)
	\1 NGC
	\2 The NGC designation of the globular cluster, or the globular cluster’s name if an NGC designation could not be found. (ex. NGC 5272, NGC 5904, NGC 6121)
	\1 Oost Type
	\2 The Oosterhoff type that the paper suggests that the globular cluster is. (I, II, III, Oo-Int, Oo-Neutral)
	\1 Source
	\2 The paper that suggests the Oosterhoff classification for the globular cluster.
	\1 Notes
	\2 Miscellaneous notes about the particular Oosterhoff type classification.
\end{outline}

\newpage

\begin{longtable}{
	p{1.5cm}|
	p{2.5cm}|
	p{2.5cm}|
	p{3.7cm}|
	p{5.5cm}
	@{}}
		\textbf{Name}           & \textbf{NGC}          & \textbf{Oost Type}  & \textbf{Source}               & \textbf{Notes}                                                                  \vspace{12pt}\\
		M3             & NGC 5272     & I          & Braga, 2016 \cite{braga_2016}         &                                                                        \\
		M5             & NGC 5904     & I          & Braga, 2016          &                                                                        \\
		M4             & NGC 6121     & I          & Braga, 2016          &                                                                        \\
		& NGC 6229     & I          & Braga, 2016          &                                                                        \\
		& NGC 6362     & I          & Braga, 2016          &                                                                        \\
		M72            & NGC 6981     & I          & Braga, 2016          &                                                                        \\
		& IC 4499      & Oo-Int     & Braga, 2016          &                                                                        \\
		& NGC 3201     & Oo-Int     & Braga, 2016          &                                                                        \\
		M54            & NGC 6715     & Oo-Int     & Braga, 2016          &                                                                        \\
		& NGC 6934     & Oo-Int     & Braga, 2016          &                                                                        \\
		& NGC 7006     & Oo-Int     & Braga, 2016          &                                                                        \\
		M68            & NGC 4590     & II         & Braga, 2016          &                                                                        \\
		M53            & NGC 5024     & II         & Braga, 2016          &                                                                        \\
		& NGC 5286     & II         & Braga, 2016          &                                                                        \\
		M15            & NGC 7078     & II         & Braga, 2016          &                                                                        \\
		& NGC 6388     & III        & Braga, 2016          &                                                                        \\
		& NGC 6441     & III        & Braga, 2016          &                                                                        \\
		& NGC 6388     & III        & Sollima, 2014 \cite{sollima_2014}        &                                                                        \\
		& NGC 6441     & III        & Sollima, 2014        &                                                                        \\
		& NGC 6229     & I          & Sollima, 2014        &                                                                        \\
		M72            & NGC 6981     & I          & Sollima, 2014        &                                                                        \\
		& NGC 6584     & I          & Sollima, 2014        &                                                                        \\
		M3             & NGC 5272     & I          & Sollima, 2014        &                                                                        \\
		& NGC 3201     & I          & Sollima, 2014        &                                                                        \\
		& NGC 6934     & I          & Sollima, 2014        &                                                                        \\
		& IC 4499      & I          & Sollima, 2014        &                                                                        \\
		M2             & NGC 7089     & II         & Sollima, 2014        &                                                                        \\
		M22            & NGC 6656     & II         & Sollima, 2014        &                                                                        \\
		& NGC 5286     & II         & Sollima, 2014        &                                                                        \\
		& NGC 4833     & II         & Sollima, 2014        &                                                                        \\
		& NGC 1466     & Oo-Int     & Sollima, 2014        &                                                                        \\
		& F2           & I          & Sollima, 2014        &                                                                        \\
		& F3           & Oo-Int     & Sollima, 2014        &                                                                        \\
		& F4           & Oo-Int     & Sollima, 2014        &                                                                        \\
		& NGC 6441     & III        & Jang, 2015 \cite{jang_2015}           &                                                                        \\
		& NGC 6388     & III        & Jang, 2015           &                                                                        \\
		& NGC 1466     & Oo-Int     & Kuehn, 2013 \cite{kuehn_2013}          &                                                                        \\
		& NGC 2210     & Oo-Int     & Kuehn, 2013          &                                                                        \\
		M3             & NGC 5272     & I          & Kuehn, 2013          &                                                                        \\
		M15            & NGC 7078     & II         & Kuehn, 2013          &                                                                        \\
		M75            & NGC 6864     & Oo-Int     & Kuehn, 2013          &                                                                        \\
		M75            & NGC 6864     & Oo-Int     & Corwin, 2003 \cite{corwin_2003}         &                                                                        \\
		& NGC 362      & I          & Sz\'ekely, 2007 \cite{szekely_2007}       &                                                                        \\
		& NGC 2419     & II         & Di Criscienzo, 2011 \cite{di_criscienzo_2011_a}  &                                                                        \\
		& NGC 1835     & Oo-Int     & Clementini, 2004 \cite{clementini_2004}     &                                                                        \\
		M15            & NGC 7078     & II         & Clementini, 2004     &                                                                        \\
		M68            & NGC 4590     & II         & Clementini, 2004     &                                                                        \\
		& IC 4499      & I          & Clement, 1991 \cite{clement_1991_a}        &                                                                        \\
		M3             & NGC 5272     & I          & Clement, 1991        &                                                                        \\
		M68            & NGC 4590     & II         & Clement, 1991        &                                                                        \\
		& NGC 2419     & II         & Clement, 1991        &                                                                        \\
		M15            & NGC 7078     & II         & Clement, 1991        &                                                                        \\
		& NGC 6426     & II         & Clement, 1991        &                                                                        \\
		M3             & NGC 5272     & I          & Smolec, 2017 \cite{smolec_2017}         &                                                                        \\
		& NGC 6362     & I          & Smolec, 2017         &                                                                        \\
		Omega Centauri & NGC 5139     & II         & Smolec, 2017         &                                                                        \\
		M62            & NGC 6266     & I          & Contreras, 2010 \cite{contreras_2010}      &                                                                        \\
		M15            & NGC 7078     & II         & Contreras, 2010      &                                                                        \\
		& NGC 6362     & I          & Contreras, 2010      &                                                                        \\
		M107           & NGC 6171     & I          & Contreras, 2010      &                                                                        \\
		M5             & NGC 5904     & I          & Contreras, 2010      &                                                                        \\
		& NGC 6229     & I          & Contreras, 2010      &                                                                        \\
		& NGC 6934     & I          & Contreras, 2010      &                                                                        \\
		M3             & NGC 5272     & I          & Contreras, 2010      &                                                                        \\
		M2             & NGC 7089     & II         & Contreras, 2010      &                                                                        \\
		& NGC 5286     & II         & Contreras, 2010      &                                                                        \\
		M55            & NGC 6809     & II         & Contreras, 2010      &                                                                        \\
		& NGC 4147     & I          & Contreras, 2010      &                                                                        \\
		& NGC 2298     & II         & Contreras, 2010      &                                                                        \\
		M68            & NGC 4590     & II         & Contreras, 2010      &                                                                        \\
		M92            & NGC 6341     & II         & Contreras, 2010      &                                                                        \\
		M9             & NGC 6333     & II         & Arellano Ferro, 2013 \cite{arellano_ferro_2013} &                                                                        \\
		M53            & NGC 5024     & II         & Arellano Ferro, 2013 &                                                                        \\
		M15            & NGC 7078     & II         & Arellano Ferro, 2013 &                                                                        \\
		M68            & NGC 4590     & II         & Arellano Ferro, 2013 &                                                                        \\
		& NGC 6388     & III        & Arellano Ferro, 2017 \cite{arellano_ferro_2017} &                                                                        \\
		& NGC 6441     & III        & Arellano Ferro, 2017 &                                                                        \\
		& NGC 1851     & I          & Arellano Ferro, 2017 &                                                                        \\
		& NGC 3201     & I          & Arellano Ferro, 2017 &                                                                        \\
		& NGC 4147     & I          & Arellano Ferro, 2017 &                                                                        \\
		M3             & NGC 5272     & I          & Arellano Ferro, 2017 &                                                                        \\
		M5             & NGC 5904     & I          & Arellano Ferro, 2017 &                                                                        \\
		M107           & NGC 6171     & I          & Arellano Ferro, 2017 &                                                                        \\
		& NGC 6229     & I          & Arellano Ferro, 2017 &                                                                        \\
		& NGC 6362     & I          & Arellano Ferro, 2017 &                                                                        \\
		& NGC 6366     & I          & Arellano Ferro, 2017 &                                                                        \\
		& NGC 6934     & I          & Arellano Ferro, 2017 &                                                                        \\
		M72            & NGC 6981     & I          & Arellano Ferro, 2017 &                                                                        \\
		& NGC 288      & II         & Arellano Ferro, 2017 &                                                                        \\
		M79            & NGC 1904     & II         & Arellano Ferro, 2017 &                                                                        \\
		M68            & NGC 4590     & II         & Arellano Ferro, 2017 &                                                                        \\
		M53            & NGC 5024     & II         & Arellano Ferro, 2017 &                                                                        \\
		& NGC 5053     & II         & Arellano Ferro, 2017 &                                                                        \\
		& NGC 5466     & II         & Arellano Ferro, 2017 &                                                                        \\
		M9             & NGC 6333     & II         & Arellano Ferro, 2017 &                                                                        \\
		M92            & NGC 6341     & II         & Arellano Ferro, 2017 &                                                                        \\
		M15            & NGC 7078     & II         & Arellano Ferro, 2017 &                                                                        \\
		M2             & NGC 7089     & II         & Arellano Ferro, 2017 &                                                                        \\
		M30            & NGC 7099     & II         & Arellano Ferro, 2017 &                                                                        \\
		& NGC 7492     & II         & Arellano Ferro, 2017 &                                                                        \\
		M3             & NGC 5272     & I          & Jeffery, 2011 \cite{jeffery_2011}        &                                                                        \\
		M15            & NGC 7078     & II         & Stetson, 2014 \cite{stetson_2014}        &                                                                        \\
		M68            & NGC 4590     & II         & Stetson, 2014        &                                                                        \\
		M22            & NGC 6656     & II         & Stetson, 2014        &                                                                        \\
		& NGC 3201     & I          & Stetson, 2014        &                                                                        \\
		& NGC 1851     & I          & Stetson, 2014        &                                                                        \\
		& NGC 4147     & I          & Stetson, 2014        &                                                                        \\
		M54            & NGC 6715     & I          & Stetson, 2014        & This and M4 could be Oo-Int depending on the classification diagnostic \\
		M4             & NGC 6121     & Oo-Neutral & Stetson, 2014        &                                                                        \\
		M5             & NGC 5904     & Oo-Neutral & Stetson, 2014        &                                                                        \\
		M9             & NGC 6333     & II         & Clement, 2001 \cite{clement_2001_a}        &                                                                        \\
		M68            & NGC 4590     & II         & Clement, 2001        &                                                                        \\
		M55            & NGC 6809     & II         & Clement, 2001        &                                                                        \\
		& NGC 6426     & II         & Clement, 2001        &                                                                        \\
		M2             & NGC 7089     & II         & Clement, 2001        &                                                                        \\
		M53            & NGC 5024     & II         & Clement, 2001        &                                                                        \\
		M92            & NGC 6341     & II         & Clement, 2001        &                                                                        \\
		M15            & NGC 7078     & II         & Clement, 2001        &                                                                        \\
		M3             & NGC 5272     & I          & Clement, 2001        &                                                                        \\
		M107           & NGC 6171     & I          & Clement, 2001        &                                                                        \\
		M5             & NGC 5904     & I          & Clement, 2001        &                                                                        \\
		& NGC 6229     & I          & Clement, 2001        &                                                                        \\
		M15            & NGC 7078     & II         & Jang, 2014 \cite{jang_2014}           &                                                                        \\
		M3             & NGC 5272     & I          & Jang, 2014           &                                                                        \\
		M68            & NGC 4590     & II         & Jang, 2014           &                                                                        \\
		& NGC 5053     & II         & Jang, 2014           &                                                                        \\
		& NGC 5466     & II         & Jang, 2014           &                                                                        \\
		& NGC 6441     & III        & Jang, 2014           &                                                                        \\
		& NGC 1851     & I          & Kunder, 2013 \cite{kunder_2013_d}         &                                                                        \\
		& NGC 362      & I          & Kunder, 2013         &                                                                        \\
		& NGC 3201     & I          & Kunder, 2013         &                                                                        \\
		M68            & NGC 4590     & II         & Kunder, 2013         &                                                                        \\
		& IC 4499      & I          & Kunder, 2013         &                                                                        \\
		Rup 106        & Ruprecht 106 & I          & Kunder, 2013         &                                                                        \\
		& NGC 5053     & II         & Kunder, 2013         &                                                                        \\
		& NGC 6934     & I          & Kunder, 2013         &                                                                        \\
		M72            & NGC 6981     & I          & Kunder, 2013         &                                                                        \\
		& NGC 7006     & I          & Kunder, 2013         &                                                                        \\
		M15            & NGC 7078     & II         & Kunder, 2013         &                                                                        \\
		Omega Centauri & NGC 5139     & II         & Kunder, 2013         &                                                                        \\
		M3             & NGC 5272     & I          & Kunder, 2013         &                                                                        \\
		& NGC 5466     & II         & Kunder, 2013         &                                                                        \\
		& NGC 6229     & I          & Kunder, 2013         &                                                                        \\
		& NGC 6426     & II         & Kunder, 2013         &                                                                        \\
		& NGC 6441     & III        & Kunder, 2013         &                                                                        \\
		& NGC 6388     & III        & Kunder, 2013         &                                                                        \\
		M62            & NGC 6266     & I          & Kunder, 2013         &                                                                        \\
		M9             & NGC 6333     & II         & Kunder, 2013         &                                                                        \\
		M92            & NGC 6341     & II         & Kunder, 2013         &                                                                        \\
		& NGC 6362     & I          & Kunder, 2013         &                                                                        \\
		M28            & NGC 6626     & I          & Kunder, 2013         &                                                                        \\
		M5             & NGC 5904     & I          & Kunder, 2013         &                                                                        \\
		& NGC 5986     & II         & Kunder, 2013         &                                                                        \\
		M4             & NGC 6121     & I          & Kunder, 2013         &                                                                        \\
		M107           & NGC 6171     & I          & Kunder, 2013         &                                                                        \\
		& NGC 6712     & I          & Kunder, 2013         &                                                                        \\
		& NGC 6723     & I          & Kunder, 2013         &                                                                        \\
		M75            & NGC 6864     & I          & Kunder, 2013         &                                                                        \\
		M53            & NGC 5024     & II         & Kunder, 2013         &                                                                        \\
		M2             & NGC 7089     & II         & Kunder, 2013         &                                                                        \\
		& NGC 2419     & II         & Kunder, 2013         &                                                                        \\
		& NGC 1851     & I          & Kunder, 2013         &                                                                        \\
		& NGC 5286     & II         & Kunder, 2013         &                                                                        \\
		& NGC 2808     & I          & Kunder, 2013         &                                                                        \\
		& NGC 4147     & I          & Kunder, 2013         &                                                                        \\
		M55            & NGC 6809     & I          & Kunder, 2013         &                                                                        \\
		M54            & NGC 6715     & I          & Kunder, 2013         &                                                                        \\
		M79            & NGC 1904     & II         & Kunder, 2013         &                                                                        \\
		M3             & NGC 5272     & I          & Pritzl, 2000 \cite{pritzl_2000_a}         &                                                                        \\
		M15            & NGC 7078     & II         & Pritzl, 2000         &                                                                        \\
		M3             & NGC 5272     & I          & Kinemuchi, 2006 \cite{kinemuchi_2006_a}      &                                                                        \\
		Omega Centauri & NGC 5139     & II         & Kinemuchi, 2006      &                                                                       
\end{longtable}

\newpage

\subsection{Reproducibility Directions}
All of the code used for the data analysis performed is available at the GitHub repository below.

\vspace{12pt}

\url{https://github.com/ExcaliburZero/rr-lyrae-oosterhoff-clustering}

\vspace{12pt}

The repository includes a Makefile, which can be used to download all of the data from the OGLE website, perform the data analysis, and generate the reports with the results.

\vspace{12pt}

The data analysis scripts were written on an Ubuntu Linux machine, so they should work on all Linux distributions and Mac. However, no guarantee as to whether they will work on a Windows machine.

\vspace{12pt}

The data analysis scripts depend on the following software, so to re-run the analysis you will need to install the following programs.

\begin{itemize}
	\item Pipenv	-	\url{https://pipenv.readthedocs.io/en/latest/}
	\item Pyenv		-	\url{https://github.com/pyenv/pyenv}
\end{itemize}

Once you have these dependencies installed, you can download and run the analysis scripts by running the following terminal commands.

\vspace{12pt}

\begin{lstlisting}[language=bash]
git clone https://github.com/ExcaliburZero/rr-lyrae-oosterhoff-clustering
pipenv install
make
\end{lstlisting}

\newpage

\includepdf[
%% Include all pages of the PDF
pages=-,
%% make this page have the usual page style
%% (you can change it to plain etc). By default pdfpages
%% sets the pagecommand to \pagestyle{empty}
%pagecommand={\pagestyle{headings}},  
pagecommand={},  
%% Add a "section" entry to the ToC with the heading
%% "Quilling Shapes" and the label "sec:shapes"
addtotoc={1,subsection,1,Data Analysis Jupyter Notebook,sec:shapes}]
%% The pdf file itself
{RRab_OGLE_IV_Clustering.pdf}

\end{document}
