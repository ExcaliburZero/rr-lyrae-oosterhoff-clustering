\documentclass[]{article}

\usepackage[a4paper, margin=1in]{geometry}

\usepackage{graphicx}
\usepackage{caption}
\usepackage{subcaption}
\usepackage{setspace}
\usepackage{hyperref}
\usepackage{outlines}
\usepackage{longtable}
\usepackage{pdflscape}
\usepackage{listings}
\usepackage{amsmath}

\usepackage[final]{pdfpages}

\newcommand{\mnras}{Monthly Notices of the Royal Astronomical Society}
\newcommand{\aap}{Astronomy and Astrophysics}
\newcommand{\aj}{The Astronomical Journal}
\newcommand{\actaa}{Acta Astronomica}
\newcommand{\apj}{Astrophysical Journal}
\newcommand{\rmxaa}{Revista Mexicana de Astronom\'ia y Astrof\'isica}
\newcommand{\apss}{Astrophysics and Space Science}

\pagenumbering{roman}

%opening
\title{}
\author{Christopher Wells}

\begin{document}

\begin{center}
	\vspace*{200pt}
	
	\textbf{Clustering Fundamental-mode RR Lyrae in the Magellanic Clouds to Determine a New Boundary Line for the Oosterhoff Dichotomy}
	
	\vspace{20pt}
	
	Christopher Wells
	
	Candidate for B.A. Degree
	
	in Computer Science
	
	\vspace{20pt}
	
	State University of New York, College at Oswego
	
	College Honors Program
	
	\vspace{20pt}
	
	December, 2018
\end{center}

\newpage

\singlespacing

\begin{abstract}
	
	RR Lyrae are a type of periodic variable star that are often examined due to their utility as distance indicators in space. An effect known as the Oosterhoff Dichotomy has been seen in populations of RR Lyrae stars in globular clusters in the Milky Way Galaxy. This phenomenon is thought to have some importance in learning more about the formation of the Milky Way Galaxy.
	
	\vspace{12pt}
	
	Previous attempts have been made to divide or fit relations to the Fundamental-mode RR Lyrae populations in globular clusters in relation to the Oosterhoff I and II classifications of Milky Way globular clusters. However, these previous attempts have often been with small datasets, or have not explained how they obtained the fits or divisions they found.
	
	\vspace{12pt}
	
	In this thesis, I examine data from the fourth data release of the Optical Gravitational Lensing Experiment (OGLE IV) survey on RR Lyrae in the Small and Large Magellanic Clouds. Using this dataset of 17,235 RRab in the LMC and 5,105 RRab in the SMC I apply kernel k-means clustering to attempt to find a partitioning of the RRab in period and amplitude that relates to the Oosterhoff dichotomy.

\end{abstract}

\newpage

\doublespacing

\tableofcontents

\newpage

\pagenumbering{arabic}

\section{Advice to Future Honors Thesis Students}

\newpage

\section{Acknowledgements}

\newpage

\section{Author's Reflections}

\newpage

\section{RR Lyrae}

RR Lyrae are a type of radially pulsating, periodic variable star that are numerous in Local Group galaxies and examined due to their utility as distance indicators and indicators of early stellar composition. \cite{szczygiel_2009, kinemuchi_2006_a}

RR Lyrae are advanced age stars ($>$ 10 Gyrs) that undergo radial pulsations, whereby they expand and contract in radius over time due to instabilities caused by radiation and gravity. \cite{szczygiel_2009, templeton_2010} These pulsations that RR Lyrae undergo cause the brightness of the star to change over time, with it becoming brighter as it expands and dimmer as it contracts. (?) This change in brightness over time classifies RR Lyrae as being variable stars, however unlike some variable stars, RR Lyrae change in brightness at a regular interval, or period, which classifies them as being periodic variable stars. \cite{AAVSO_nodate}

RR Lyrae are numerous in the Milky Way and other Local Group galaxies and are additionally easy to identify. \cite{soszynski_2016, kinemuchi_2006} RR Lyrae are often researched due to their use as ``standard candles'' (distance indicators). \cite{soszynski_2016} The intensity of the luminosity change caused by the periodic pulsations is related with the period of the pulsations, thus the period can be used to calculate the luminosity of the star which can be compared with the observed brightness of the star to determine its distance. \cite{templeton_2010}

\subsection{Light Curves}

When examining variable stars, such as RR Lyrae, we often are working with data on the brightness of the variable star over a given period of time. Sky surveys that gather data on variable stars take photometric brightness measurements of many different variable stars at various different points in time. These brightness measurements over time for a particular variable star are called a light curve. \cite{AAVSO_nodate}

By examining the light curve of a variable star we can get an idea of the type of brightness variation that the variable star is undergoing. Different types of variable stars tend to have different characteristic patterns of brightness variation. \cite{sokolovsky_2017}

Light curves are typically plotted as a scatter plot where the x-axis is the time of the observation in Modified Julian Date (MJD) and the y-axis is the brightness of the star in magnitude. \cite{AAVSO_nodate}

In the case of periodic variable stars, like RR Lyrae, a transformation called period folding is often applied to the light curve to look specifically at the periodic variation that the star undergoes. To period fold a light curve, the time for each brightness measurement is ``folded'' over the period of the star by taking the remainder of the division of the time by the period to get the ``phase'' of the observation. \footnote{For non-periodic variable stars, period folding of the light curve is not possible, since the star does not have a variation period to fold the observations over.} \cite{AAVSO_nodate}

This ``period folded light curve'' makes it easier to see the pattern of variation that a periodic variable star undergoes, since some periodic variable stars undergo variations that have a higher frequency than the frequency at which the survey observes them. \footnote{``Period folded light curves'' are also referred to as ``folded light curves'' or ``phase diagrams''. \cite{AAVSO_nodate}} For such periodic variable stars it is difficult to see the variation pattern as each instance of the pattern may have only a few observations. However, when the light curve is period folded, the different instances of the variation pattern line up over one another, making the repeated pattern more apparent. So the low sampling of each instance of the pattern is countered by the high sampling of all of the instances of the pattern as a whole. \cite{AAVSO_nodate}

\begin{figure}
	\centering
	\begin{subfigure}{.5\textwidth}
		\centering
		\includegraphics[width=6cm]{figures/light_curve_examples/light_curve_raw_i.png}
		\label{fig:light_curve_raw_i}
	\end{subfigure}%
	\begin{subfigure}{.5\textwidth}
		\centering
		\includegraphics[width=6cm]{figures/light_curve_examples/light_curve_folded_i.png}
		\label{fig:sub2}
	\end{subfigure}
	\caption{Plots of the light curve (left) and period folded light curve (right) for the RRab (OGLE-LMC-RRLYR-00001) from the OGLE IV LMC dataset. In the light curve (left) the variance in the RRab’s brightness over time is visible, however, it is only in the period folded light curve (right) that the pattern of the variability is clearly shown.}
	\label{fig:light_curves}
\end{figure}

\subsection{Properties \& Sub-Types}

RR Lyrae are characterized mainly through a few different properties, both physical and properties of their light curves that develop as a result of physical properties. (?) One of the most commonly examined properties of RR Lyrae is their period, which is the time interval at which they repeat their pulsations and variance in brightness. (?) RR Lyrae typically have periods of 0.2 to 1.2 days. \cite{szczygiel_2009} For example if an RR Lyrae with a period of 0.58 days reached its peak brightness, at 0.58 days later it would again be at its peak brightness.

Another commonly examined property is amplitude, or the amount of change that the brightness of the star undergoes as a part of its regular pattern. (?) RR Lyrae typically have amplitudes of 0.2 to 1.6 mag in V-band photometric band light. \cite{szczygiel_2009} Amplitude is typically calculated by taking the difference between the maximum and minimum measured brightness of the star in magnitudes. \cite{richards_2011}

$$
Amplitude = m_{max} - m_{min}
$$

Another commonly examined property is metallicity, also referred to as [Fe/H], or the ratio of iron to hydrogen in the star. \cite{jurcsik_1995} While metallicity can be measured through spectroscopy, for dealing with metallicity for large formations containing RRab estimation formulas are typically used. Some of the metallicity estimation formulas used are from Jurcsik and Kovacs (1996), Sandage (2004), Smolec (2005), and Nemec et al. (2013). \cite{jurcsik_1996, sandage_2004, smolec_2005, nemec_2013} For example, the following formula from Jurcsik and Kovacs (1996) estimates the metallicity of an RRab using its period and $\phi_{31}$. \cite{jurcsik_1996}

$$
[Fe/H] = -5.038 - 5.394 P + 1.345 \phi_{31}
$$

Where $P$ is the period and $\phi_{31}$ is defined as follows. \cite{bhardwaj_2018}

$$
\phi_{31} = \phi_{3} - 3\phi_{1}
$$

Where the $\phi_{n}$ values are the $n$th $\phi$ coefficient of the following fourier fit.

$$
magnitude = A_{0} + \sum_{k=1}^{m} A_k * sin(2\pi * k * time + \phi_{k})
$$

There are several different sub-types of RR Lyrae that tend to have different characteristics due to the mode(s) that they pulsate in. \cite{chen_2013} RR Lyrae that pulsate in the fundamental mode are known as RRab. RR Lyrae that pulsate in the first overtone mode are known as as RRc. RR Lyrae that pulsate in both the fundamental and first overtone modes are known as RRd. \footnote{An additional sub-type of RR Lyrae called RRe, which had second overtone mode pulsations, was thought to exist. However, the existence of RRe has been debated in the literature. \cite{chen_2013} For the OGLE IV data release, OGLE has stopped classifying stars as RRe, choosing to group in all stars previously classified as RRe to be RRc, so as to not separate the first and second overtone pulsators. They noted that they did this ``because of the doubts whether the RRe stars exist at all''. \cite{soszynski_2016}} \footnote{An alternate naming scheme using numbers is sometimes used. In this scheme RRab are called RR0, RRc are called RR1, RRd are called RR01, and RRe are called RR2. This system was introduced in Alcock et al. (2000) in order to allow for easier mnemonic memorization of the different types, with the 0 indicating fundamental mode, 1 indicating first overtone mode, and 2 indicating second overtone mode. \cite{chen_2013} I have decided to use the letter-based naming convention instead due to OGLE's use of it and since it seem to be the more popular naming scheme in the literature.} \cite{chen_2013} Additionally, these different type of RR Lyrae have different ranges of values for their properties like period and amplitude.


\begin{center}
\begin{tabular}{|l|c|c|c|c|}
	\hline
	Sub-type & Mean FM Period & Mean FM I-band Amplitude & Mean FO Period & Mean FO I-band Amplitude \\
	\hline
	RRab & 0.58 & 0.53 & - & - \\
	RRc & - & - & 0.33 & 0.26 \\
	RRd & 0.49 & 0.16 & 0.36 & 0.26 \\
	\hline
\end{tabular}
\captionof{table}{Summary information on the properties of the RR Lyrae sub-types calculated using the data from the OGLE IV data release. \cite{soszynski_2016} Column names containing ``FM'' refer to the fundamental mode pulsations and column names containing ``FO'' refer to the first overtone mode pulsations. A dash indicates that a column is not applicable for a particular sub-type.}
\end{center}

\newpage

\section{Oosterhoff Dichotomy}

In 1939, Pieter Oosterhoff published a paper on some observations he made about RR Lyrae in various globular clusters in the Milky Way galaxy. Oosterhoff noticed that if you examined the mean periods of the RRab and RRc in globular clusters, the globular clusters appeared to cluster into two groups. He found that M5 and M3 had lower RRab and RRc periods, while M53, M15, and $\omega$ Cen had higher RRab and RRc periods. \footnote{Globular clusters are typically referred to either by their NGC identifier (like NGC 5024), an M followed by a number (like M53), or a specific name (like $\omega$ Cen). (?) See the section \ref{sec:app_globular_clusters_summary} \nameref{sec:app_globular_clusters_summary} for a list of several globular clusters in the Milky Way galaxy and their various names and identifiers.} \cite{oosterhoff_1939} These two groups became known as the Oostherhoff I (OoI) and Oosterhoff II (OoII) groups. \cite{clement_2000}

\begin{figure}
	\centering
	\includegraphics[width=10cm]{figures/globular_clusters/oosterhoff_1939.png}
	\caption{A plot of the Milky Way globular clusters examined in Oosterhoff (1939). \cite{oosterhoff_1939} A gap is visible between the clusters of M5 + M3 and M53 + M15 + $\omega$ Cen.}
	\label{fig:oosterhoff_1939_globular_clusters}
\end{figure}

While Oosterhoff originally examined globular clusters in terms of the mean and median RRab and RRc periods, research since has looked at other parameters as well in terms of the Oosterhoff dichotomy including metalicity and color. \cite{clement_2000}

However, examinations of globular clusters outside of the Milky Way found some globular clusters, such as those in the dwarf spheroidal satellite galaxies of the Milky way, that exist in between these two groups, in the ``Oosterhoff gap''. \cite{szczygiel_2009, catelan_2009, cusano_2016, joo_2018} Globular clusters that fell in between the Oosterhoff I and Oosterhoff II groups became known as Oosterhoff intermediate (Oo-Int) globular clusters. (?)

The Oosterhoff dichotomy is believed to be related to the formation of the Milky Way halo. \cite{jang_2015} In particular, it has been argued that the presence of the Oostherhoff dichotomy in Milky Way globular clusters suggests that the Milky Way halo was not formed by the accretion of ``protogalactic fragments'' like the dwarf spheroidal satellite galaxies of the Milky Way were formed. \cite{catelan_2009}

\begin{figure}
	\centering
	\includegraphics[width=17cm]{figures/globular_clusters/globular_clusters_by_oosterhoff_type.png}
	\caption{A plot of modern data on globular clusters showing the Oosterhoff classifications for each that are agreed upon in the literature. Shows both Milky Way and non-Milky Way globular clusters. ``Conflicted'' indicates that multiple different classifications have been suggested in different papers. (see section ~\ref{sec:app_globular_clusters_summary}~ \nameref{sec:app_globular_clusters_summary})}
	\label{fig:modern_globular_clusters}
\end{figure}

\begin{figure}
	\centering
	\includegraphics[width=17cm]{figures/globular_clusters/globular_clusters_by_location.png}
	\caption{A plot of the modern data on globular clusters with agreed upon Oosterhoff classifications, broken up into Milky Way and non-Milky Way globular clusters. Shows that the plotted Oosterhoff Intermediate clusters are all not in the Milky Way galaxy. (see section ~\ref{sec:app_globular_clusters_summary}~ \nameref{sec:app_globular_clusters_summary})}
	\label{fig:modern_globular_clusters_location}
\end{figure}

\subsection{Previous fits and clustering of RRab}
There has been some previous research that has looked for fit lines and divisions within the RRab stars in globular clusters that relate to the Oosterhoff dichotomy.


In 2005, Cacciari et al. used BVI banded RRab data from the OoI globular cluster M3 to fit a quadratic relation between log period and B-band amplitude. \cite{cacciari_2005}

$$
A_{B} = -3.123 - 26.331~log(P) - 35.853~log(P^{2})
$$

In 2009, Szczygie{\l} et al. used V-band RRab data from the ASAS survey to fit a piecewise linear function and a linear for Oosterhoff I and II subgroups, respectively, based on log period and V-band amplitude. They also determine a linear function to separate the Oosterhoff I and II groups. \cite{szczygiel_2009}

$$
(OoI)~A_{v} =
\begin{cases} 
	-8.844 * log(P) - 1.343 & A_{v} < 1.2 \\
	-1.654 * log(P) + 0.719 & A_{v} \geq 1.2 \\
\end{cases}
$$

$$
(OoII)~A_{v} = -7.007 * log(P) - 0.343
$$
$$
(Separation)~A_{v} = -8.0 * log(P) - 0.85
$$

In 2013, Kunder et al. provide a linear division of ``OoI-type'' RRab and ``OoII-type'' RRab based on period and V-band amplitude. For this division, the OoI-type RRab fall above the following line, and the Oo-I-type RRab fall below it. \cite{kunder_2013_d}

$$
A_{v} = -5.1453~P + 4.02
$$

\newpage

\section{Clustering}

Clustering is the process of looking for patterns groupings of points or patterns in a dataset, typically though the use of automated clustering algorithms. The idea is to find partitions of the data into clusters such that intercluster distance is minimized, or rather the in group similarity should be high. Additionally the intracluster distance should be maximized, or rather the similarity between things in different clusters should be low. \cite{jain_2010}

By applying clustering algorithms to a dataset, we can learn more about sub-categories that exist in the data. \cite{jain_2010}

\subsection{K-Means Clustering}

The clustering algorithm that I used for the data analysis I performed was K-Means clustering. K-Means clustering is an algorithm that takes as input a number of clusters $k$ to try to partition the data into and works by making initial guesses for the $k$ clusters and gradually refining them until a convergence threshold or iteration limit is reached. \footnote{There are other clustering algorithms that do not require the number of clusters to be specified before running the algorithm, some decide the number of clusters on their own based on the data. I choose K-Means because in the use case I had I was reasonably sure the number of clusters would be 2.} \cite{jain_2010}

There are a few different algorithms for K-Means clustering, but one common one is Lloyd’s algorithm. \cite{arthur_2007}

\begin{enumerate}
	\item Choose $k$ different random data points for the dataset to be the centers, or ``centroids'', of the clusters.
	\item Until an iteration limit or convergence is reached, repeat the following:
	\begin{enumerate}
		\item Assign each point in the dataset to the cluster of the centroid that it is closest to.
		\item For each cluster:
		\begin{enumerate}
			\item Compute a new ``centroid'' for the cluster by calculating the ``center'' of all of the data points in the cluster. \footnote{Note that this calculated new ``centroid'' point is most often not an actual data point in the dataset, but rather an abstract theoretical point.}
		\end{enumerate}
		\item Calculate the sum squared distances between each data point and its associated centroid.
	\end{enumerate}
\end{enumerate}

This algorithm works to continually reduce the sum of squared distances of the points within each cluster from their associated centroid, defined by the following equation. \cite{arthur_2007}

$$
\sum_{x \in \chi}^{}\min_{c \in C} || x - c ||^2
$$

Where $\chi$ is the collection of data points, $C$ is the collection of centroids, and $c$ is the centroid closest to the data point $x$.

\subsection{K-Means Implementation used}
For the data analysis, I used the implementation of K-Means clustering in the Scikit-learn Python library. \cite{pedregosa_2011} This implementation uses either Lloyd’s algorithm or Elkan’s algorithm depending on whether the input data is dense or sparse. \cite{scikit_learn_kmeans} Since the data I worked with was dense, Scikit-learn used Elkan’s algorithm.

Elkan’s algorithm is an optimized algorithm for K-Means that makes use of the triangle inequality to avoid unnecessary distance calculations. It works by using the triangle inequality to get upper and lower bounds on the distances between points and centroids to avoid calculating the actual distances, as only the closest centroid to each data point needs to found, so calculating the actual distances is not necessary. \cite{elkan_2003}

\subsection{Kernel functions \& Kernel K-Means}

Kernel functions are transformation functions that can be applied in classification and clustering algorithms, such as support vector machines, to allow for easier classification and clustering of nonlinear patterns. Kernel functions are applied to the input data of an algorithm to transform it from the original feature space into a new feature space in which subgroups within the data are more easily separable. \cite{amari_1999} Kernel functions can also be applied to some clustering algorithms like k-means to achieve clusters of arbitrary shapes. \cite{scholkopf_1998, jain_2010}

Regular k-means clustering cannot separate data into clusters that are not linearly separable. One approach for fitting arbitrary shaped clusters is the use of a kernel function within the k-means algorithm. This technique is referred to as kernel k-means clustering. Kernel k-means clustering works the same as regular k-means clustering, however when performing the distance calculations of a data point from a centroid and determining new centroids, the kernel function is applied to the data points to transform them from the input space to another feature space. \cite{dhillon_2004}

Thus kernel k-means works to iteratively minimize the sum of squared distances of the kernel function transformed data points within each cluster from their associated centroid.

$$
\sum_{x \in \chi}^{}\min_{c \in C} || \phi(x) - c ||^2
$$

Where $\chi$ is the collection of data points, $C$ is the collection of centroids, $\phi$ is the kernel function, and $c$ is the centroid closest to $\phi(x)$.


\newpage

\section{Data Analysis}

%To learn more about the Oosterhoff dichotomy I wanted to try applying clustering algorithms to the fairly recent OGLE IV dataset to see if I could separate the RRab into two clusters that have properties that correspond to the two main Oosterhoff groups.

\subsection{OGLE IV Dataset}

The Optical Gravitational Lensing Experiment (OGLE) is a long-running photometric survey that has taken CCD observations of the Galactic Bulge and the Small and Large Magellanic Clouds. \cite{udalski_1992, soszynski_2016} As of the time of writing, the most recent phase of the OGLE survey that data has been released for is the fourth phase (OGLE IV). \cite{soszynski_2016}

OGLE IV is a dataset that includes data on RR Lyrae present in the Large and Small Magellanic Clouds. The Large and Small Magellanic clouds, also referred to as the LMC and SMC respectively, are two galaxies in the Local Group that are close the Milky Way galaxy. \cite{harvard_2007}

The OGLE IV dataset contains data on a total of 45,451 RR Lyrae, 39,082 in the LMC and 6,369 in the SMC. With 17,235 RRab in the LMC and 5,105 RRab in the SMC. \cite{soszynski_2016}

The OGLE IV dataset contains light curves for the RR Lyrae in both I and V photometric bands. The I band light curves are highly sampled over time, while the V band light curves are more sparse. \cite{soszynski_2016}

\begin{center}
	\begin{tabular}{|l|l|l|l|}
		\hline
		\multicolumn{4}{|c|}{LMC} \\
		\hline
		Band & Num RR Lyrae & Mean \# Observations & Std Dev Observations \\
		\hline
		I & 39,609 & 491 & 216 \\
		V & 37,169 & 100 & 79 \\
		\hline
		\multicolumn{4}{|c|}{SMC} \\
		\hline
		Band & Num RR Lyrae & Mean \# Observations & Std Dev Observations \\
		\hline
		I & 6,560 & 404 & 150 \\
		V & 6,308 & 37 & 32 \\		
		\hline
	\end{tabular}
	\captionof{table}{OGLE IV Light Curve Observation Count Summaries}
\end{center}

\subsection{Data Selection}
To try to apply clustering algorithms to the OGLE IV dataset to try to find a partitioning of the data related to the Oosterhoff dichotomy I first needed to decide what data in the dataset to work with.

I wanted to come up with a method that would work for both of the Magellanic clouds, so I needed to work with the data for both.

The dataset contains data on several different sub-types of RR Lyrae, so I needed to decide which types to work with. Clustering with all of the different sub-types at once would not work, since they have very different values for period and pulsations in different modes. Since RRab have been looked at in previous literature and are the most numerous in the dataset, I worked with just the RRab stars in the dataset.

The dataset also contains not only the light curves of the RR Lyrae, but also some pre-calculated parameters of the stars. Since previous literature has looked at period and amplitude, I decided to look at these parameters as well. However, since the I-band data for the OGLE IV dataset is less sparsely sampled than the V-band data I worked with the I-band amplitude, unlike most previous literature which looked mainly at V-band and B-band amplitudes.

So I had decided to look at the data for the LMC and SMC in turn, for each looking at the period and I-band amplitude for only the RRab stars.

\subsection{Applying Clustering}
Looking at the data, I noticed that there looked like there were two subpopulations that seemed to follow a trend in period and I-band amplitude. Additionally the two groups seemed to be mostly separable by a boundary that follows a similar shape to the overall trend in the data. So I figured that it would be best to try using a clustering technique that would fit clusters based on a boundary with such a shape.

Many clustering techniques like K-Means clustering tend to fit ``blob-like'' clusters, so I figured that using such a technique on its own would not be the best approach due to the apparent shape of the groups having a ``non-blob'' shape. So I decided to try taking an approach using a kernel function, where I would pre-transform the data points into a feature space where clustering algorithms would be able to more easily fit a boundary line of a particular shape.

In order to determine the transformation to apply to the data before using the clustering algorithm, I fit a polynomial trend line to the data. At first I tried fitting the trend line to all of the RRab in a given cluster, but the resulting trend line seemed off.

In order to get a better fit of the trend of the data I filtered it down to ignore some of the more outlying points, by using a Gaussian kernel density estimate to find what regions of the period, I-band amplitude space were more densely populated. I then applied a density threshold to look at the RRab data points in the denser region of the space. I then applied a least squares fit of a 3rd degree polynomial to fit the trend of the data.

With this trend line I could then construct a kernel function to apply to the data. The kernel function would take a point in the period, I-band amplitude space and map it to a 1 dimensional point by taking the  difference between the actual I-band amplitude value and the amplitude value predicted by the trend for the point's period value. In this 1 dimensional, trend subtracted, space the two subgroups are closer to being linearly separable.

After transforming the data to the trend subtracted space, I started applying K-Means clustering. Initially I used a $k$ value of 2, since there appeared to be two subpopulations, however some of the more outlying points, perhaps caused by the trend line having odd behavior at its ends, seemed to throw off the clustering. So instead I tried apply K-Means clustering with 3 clusters and this resulted in one cluster that seemed to fit the subpopulation that had lower period and amplitude values and two clusters that together seemed to fit the subpopulation that had higher period and amplitude values.

I then was able to examine the values in the trend subtracted space at which the K-Means clustering separated the ``lower'' cluster from the ``upper'' clusters. Then by using the kernel function to map this linear boundary line back to the period, I-band amplitude space I was able to get a 3rd degree polynomial for separating the two groups in period and I-band amplitude.

For a full breakdown of the code used in the analysis, the intermediate plots of the data, and the equations for the fitted boundary lines see Section \ref{sec:data-analysis} \nameref{sec:data-analysis}.

Then using these boundary lines for the SMC and LMC I was able to take the mean RRab period and metalicity (estimated using the Jurcsik and Kovacs formula) of the two subpopulations for each of the two Magellanic clouds. These summary data points seemed to line up with the Oosterhoff I and II group globular clusters examined in the literature. (See Figures \ref{fig:lmc_clustering_with_gcs} and \ref{fig:smc_clustering_with_gcs})

[add image of trend subtracted space]

[add image of the final clusters]

\begin{figure}
	\centering
	\includegraphics[width=17cm]{figures/globular_clusters/lmc_clusters_with_globular_clusters.png}
	\caption{The position of the mean values of the two clusters in the LMC data, plotted against globular clusters to show their relation to the Oosterhoff classifications.}
	\label{fig:lmc_clustering_with_gcs}
\end{figure}

\begin{figure}
	\centering
	\includegraphics[width=17cm]{figures/globular_clusters/smc_clusters_with_globular_clusters.png}
	\caption{The position of the mean values of the two clusters in the SMC data, plotted against globular clusters to show their relation to the Oosterhoff classifications.}
	\label{fig:smc_clustering_with_gcs}
\end{figure}

\subsection{Analyzing results}
While I was able to apply clustering to determine lines to separate the RRab in the LMC and SMC into subpopulations that seem to relate to the Oosterhoff dichotomy, I wanted to see if these “boundary lines” have a similar effect when applied to the RRab in various globular clusters.

The idea being that if I could apply the boundary line to globular clusters with known Oosterhoff classifications, the ratio of RRab on each side of the boundary line should be related to the Oosterhoff classification of the globular cluster.

To figure this out, I started looking on VizieR and AdsAbs for data releases of RR Lyrae data from globular clusters. \footnote{The VizieR Catalog is an online database of astronomical catalogues which allows for querying and downloading the data of a number of different data releases. \cite{ochsenbein_2000}} \footnote{The SAO/NASA Astrophysics Data System is a searchable database of papers for fields related to astronomy and astrophysics. \cite{accomazzi_2015}} I recorded each of the relevant data releases that I found into a spreadsheet. I was able to find a total of 41 different data releases. (See section \ref{sec:app_data_releases} \nameref{sec:app_data_releases})

However, I needed to filter down the data releases to the ones that I would be able to use to evaluate my clustering results. Since I was working with period and I-band amplitude I would only be able to work with data releases that had I-band photometry. Out of the 41 data releases I found, 18 had I-band photometry.

I also wanted to only look at globular clusters that had known and agreed upon Oosterhoff classifications, since otherwise I would not be able to evaluate how the boundary line relates to the globular cluster’s Oosterhoff classification. Out of the 18 remaining data releases, 12 were for globular clusters with agreed upon Oosterhoff classifications.

I also wanted to look at different globular clusters, so in the case of having multiple data releases for the same globular cluster I would only use one of the available data releases. With the 12 remaining data releases, there were 8 different globular clusters with data.

I also wanted to look at globular clusters with differing Oosterhoff classifications to get a good idea of how the boundary line relates to the different Oosterhoff classifications. However, of the 8 globular clusters that I had found data releases for 7 of them were OoII and only one was OoI.

Unfortunately only having one OoI globular cluster would not be enough to get a good idea of how the boundary line that I found relates to the Oosterhoff classifications. So I was unable to do this analysis.

\newpage

\singlespacing

\section{References}

\begingroup
\renewcommand{\section}[2]{}
\bibliographystyle{plain}
\bibliography{references}
\endgroup

\newpage

\section{Appendices}

\subsection{Collected Globular Cluster Information}

\subsubsection{Globular Clusters Summary}
\label{sec:app_globular_clusters_summary}

This table contains information on various globular clusters, including information such as their location, suggested Oosterhoff classifications, and various summary statistics on their RRab and RRc.

\vspace{12pt}

Each row in the table contains information on one particular globular cluster.

\vspace{12pt}

This data was collected by examining different papers related to the Oosterhoff dichotomy and collecting data that was listed in various tables about globular clusters. This information was then put into a spreadsheet and filtered down to globular clusters that met the following inclusion criteria.

\begin{itemize}
	\item I was able to find at least one suggested Oosterhoff classification
	\item The Oosterhoff classification was not ``mixed''
	\item I was able to find information about the location of the globular cluster
\end{itemize}

\textbf{Link:} \url{https://github.com/ExcaliburZero/rr-lyrae-oosterhoff-clustering/blob/master/data/raw/gc_oosterhoff/Collected\%20Globular\%20Cluster\%20Information\%20-\%20Globular\%20Clusters\%20Summary.csv}

\paragraph{Columns}

\begin{outline}
	\1 Name
	\2 The non-NGC name of the globular cluster. (if any) (ex. M3, M5, M72)
	\1 NGC
	\2 The NGC designation of the globular cluster, or the globular cluster’s name if an NGC designation could not be found. (ex. NGC 5272, NGC 5904, NGC 6121)
	\1 GC Location
	\2 The location of the globular cluster. (ex. Milky Way, LMC, Fornax dSph, ...)
	\1 GC Location Source
	\2 The paper that mentions the location of the globular cluster.
	\1 Num Oost Types
	\2 The number of different Oosterhoff classifications that were mentioned by papers for the globular cluster.
	\1 Num Oost Sources
	\2 The number of different papers that mentioned an Oosterhoff classification for the globular cluster.
	\1 Oost Type
	\2 The Oosterhoff classification mentioned in papers for the globular cluster. (ex. I, II, III, Oo-Int, Conflicted)
	\2 ``Conflicted'' indicates that more than one Oosterhoff classification has been suggested for the globular cluster.
	\1 Mean RRab Period
	\2 The mean period of known RRab stars in the globular cluster.
	\1 Mean RRab Period Source
	\2 The paper that mentions the ``Mean RRab Period'' of the globular cluster.
	\1 Mean RRc Period
	\2 The mean period of known RRc stars in the globular cluster.
	\1 Mean RRc Period Source
	\2 The paper that mentions the ``Mean RRc Period'' of the globular cluster.
	\1 Metalicity
	\2 The iron abundance/metallicity of the globular cluster.
	\1 Metalicity Source
	\2 The paper the mentions the ``Metallicity'' of the globular cluster.
	\1 \# RRab
	\2 The number of known RRab stars in the globular cluster.
	\1 \# RRab Source
	\2 The paper that mentions the ``\# RRab''.
\end{outline}

\paragraph{Sources}

\begin{itemize}
	\item Arellano Ferro, 2013 \cite{arellano_ferro_2013}
	\item Arellano Ferro, 2016 \cite{arellano_ferro_2016_b}
	\item Braga, 2016 \cite{braga_2016}
	\item Clement, 2001 \cite{clement_2001_a}
	\item Clementini, 2004 \cite{clementini_2004}
	\item Contreras, 2010 \cite{contreras_2010}
	\item Jang, 2015 \cite{jang_2015}
	\item Kains, 2015 \cite{kains_2015}
	\item Kuehn, 2013 \cite{kuehn_2013}
	\item Kunder, 2013 \cite{kunder_2013_d}
	\item Sollima, 2014 \cite{sollima_2014}
	\item Soszyński, 2016 \cite{soszynski_2016}
	\item Sources for Oosterhoff classifications are listed in section~\ref{sec:app_oosterhoff_classifications}~\nameref{sec:app_oosterhoff_classifications}.
\end{itemize}

\newpage

\subsubsection{Oosterhoff Classifications}
\label{sec:app_oosterhoff_classifications}

This table contains information on mentions of the Oosterhoff classifications of various globular clusters. This information was used to create the ``Num Oost Types'', ``Num Oost Sources'', and ``Oost Type'' columns in the ``Globular Clusters Summary'' table.

\vspace{12pt}

Additionally this table shows that there are some globular clusters that have had multiple different Oosterhoff classifications suggested by different papers in the literature.

\vspace{12pt}

This data was collected by looking through papers related to the Oosterhoff dichotomy and looking for mentions of the Oosterhoff classifications of globular clusters in the text of the paper or in tables in the paper. This information was then put into a spreadsheet and filtered down to globular clusters that met the same inclusion criteria as the ``Globular Clusters Summary'' table.

\vspace{12pt}

\textbf{Link:} \url{https://github.com/ExcaliburZero/rr-lyrae-oosterhoff-clustering/blob/master/data/raw/gc_oosterhoff/Collected\%20Globular\%20Cluster\%20Information\%20-\%20Oosterhoff\%20Classifications.csv}

\paragraph{Columns}

\begin{outline}
	\1 Name
	\2 The non-NGC name of the globular cluster. (if any) (ex. M3, M5, M72)
	\1 NGC
	\2 The NGC designation of the globular cluster, or the globular cluster’s name if an NGC designation could not be found. (ex. NGC 5272, NGC 5904, NGC 6121)
	\1 Oost Type
	\2 The Oosterhoff type that the paper suggests that the globular cluster is. (I, II, III, Oo-Int, Oo-Neutral)
	\1 Source
	\2 The paper that suggests the Oosterhoff classification for the globular cluster.
	\1 Notes
	\2 Miscellaneous notes about the particular Oosterhoff type classification.
\end{outline}

\paragraph{Sources}

\begin{itemize}
	\item Arellano Ferro, 2013 \cite{arellano_ferro_2013}
	\item Arellano Ferro, 2017 \cite{arellano_ferro_2017}
	\item Braga, 2016 \cite{braga_2016}
	\item Clement, 1991 \cite{clement_1991_a}
	\item Clement, 2001 \cite{clement_2001_a}
	\item Clementini, 2004 \cite{clementini_2004}
	\item Contreras, 2010 \cite{contreras_2010}
	\item Corwin, 2003 \cite{corwin_2003}
	\item Di Criscienzo, 2011 \cite{di_criscienzo_2011_a}
	\item Jang, 2014 \cite{jang_2014}
	\item Jang, 2015 \cite{jang_2015}
	\item Jeffery, 2011 \cite{jeffery_2011}
	\item Kinemuchi, 2006 \cite{kinemuchi_2006_a}
	\item Kuehn, 2013 \cite{kuehn_2013}
	\item Kunder, 2013 \cite{kunder_2013_d}
	\item Pritzl, 2000 \cite{pritzl_2000_a}
	\item Smolec, 2017 \cite{smolec_2017}
	\item Sollima, 2014 \cite{sollima_2014}
	\item Stetson, 2014 \cite{stetson_2014}
	\item Székely, 2007 \cite{szekely_2007}
\end{itemize}

\newpage

\begin{longtable}{
	p{1.5cm}|
	p{2.5cm}|
	p{2.5cm}|
	p{3.7cm}|
	p{5.5cm}
	@{}}
		\textbf{Name}           & \textbf{NGC}          & \textbf{Oost Type}  & \textbf{Source}               & \textbf{Notes}                                                                  \vspace{12pt}\\
		M3             & NGC 5272     & I          & Braga, 2016 \cite{braga_2016}         &                                                                        \\
		M5             & NGC 5904     & I          & Braga, 2016          &                                                                        \\
		M4             & NGC 6121     & I          & Braga, 2016          &                                                                        \\
		& NGC 6229     & I          & Braga, 2016          &                                                                        \\
		& NGC 6362     & I          & Braga, 2016          &                                                                        \\
		M72            & NGC 6981     & I          & Braga, 2016          &                                                                        \\
		& IC 4499      & Oo-Int     & Braga, 2016          &                                                                        \\
		& NGC 3201     & Oo-Int     & Braga, 2016          &                                                                        \\
		M54            & NGC 6715     & Oo-Int     & Braga, 2016          &                                                                        \\
		& NGC 6934     & Oo-Int     & Braga, 2016          &                                                                        \\
		& NGC 7006     & Oo-Int     & Braga, 2016          &                                                                        \\
		M68            & NGC 4590     & II         & Braga, 2016          &                                                                        \\
		M53            & NGC 5024     & II         & Braga, 2016          &                                                                        \\
		& NGC 5286     & II         & Braga, 2016          &                                                                        \\
		M15            & NGC 7078     & II         & Braga, 2016          &                                                                        \\
		& NGC 6388     & III        & Braga, 2016          &                                                                        \\
		& NGC 6441     & III        & Braga, 2016          &                                                                        \\
		& NGC 6388     & III        & Sollima, 2014 \cite{sollima_2014}        &                                                                        \\
		& NGC 6441     & III        & Sollima, 2014        &                                                                        \\
		& NGC 6229     & I          & Sollima, 2014        &                                                                        \\
		M72            & NGC 6981     & I          & Sollima, 2014        &                                                                        \\
		& NGC 6584     & I          & Sollima, 2014        &                                                                        \\
		M3             & NGC 5272     & I          & Sollima, 2014        &                                                                        \\
		& NGC 3201     & I          & Sollima, 2014        &                                                                        \\
		& NGC 6934     & I          & Sollima, 2014        &                                                                        \\
		& IC 4499      & I          & Sollima, 2014        &                                                                        \\
		M2             & NGC 7089     & II         & Sollima, 2014        &                                                                        \\
		M22            & NGC 6656     & II         & Sollima, 2014        &                                                                        \\
		& NGC 5286     & II         & Sollima, 2014        &                                                                        \\
		& NGC 4833     & II         & Sollima, 2014        &                                                                        \\
		& NGC 1466     & Oo-Int     & Sollima, 2014        &                                                                        \\
		& F2           & I          & Sollima, 2014        &                                                                        \\
		& F3           & Oo-Int     & Sollima, 2014        &                                                                        \\
		& F4           & Oo-Int     & Sollima, 2014        &                                                                        \\
		& NGC 6441     & III        & Jang, 2015 \cite{jang_2015}           &                                                                        \\
		& NGC 6388     & III        & Jang, 2015           &                                                                        \\
		& NGC 1466     & Oo-Int     & Kuehn, 2013 \cite{kuehn_2013}          &                                                                        \\
		& NGC 2210     & Oo-Int     & Kuehn, 2013          &                                                                        \\
		M3             & NGC 5272     & I          & Kuehn, 2013          &                                                                        \\
		M15            & NGC 7078     & II         & Kuehn, 2013          &                                                                        \\
		M75            & NGC 6864     & Oo-Int     & Kuehn, 2013          &                                                                        \\
		M75            & NGC 6864     & Oo-Int     & Corwin, 2003 \cite{corwin_2003}         &                                                                        \\
		& NGC 362      & I          & Sz\'ekely, 2007 \cite{szekely_2007}       &                                                                        \\
		& NGC 2419     & II         & Di Criscienzo, 2011 \cite{di_criscienzo_2011_a}  &                                                                        \\
		& NGC 1835     & Oo-Int     & Clementini, 2004 \cite{clementini_2004}     &                                                                        \\
		M15            & NGC 7078     & II         & Clementini, 2004     &                                                                        \\
		M68            & NGC 4590     & II         & Clementini, 2004     &                                                                        \\
		& IC 4499      & I          & Clement, 1991 \cite{clement_1991_a}        &                                                                        \\
		M3             & NGC 5272     & I          & Clement, 1991        &                                                                        \\
		M68            & NGC 4590     & II         & Clement, 1991        &                                                                        \\
		& NGC 2419     & II         & Clement, 1991        &                                                                        \\
		M15            & NGC 7078     & II         & Clement, 1991        &                                                                        \\
		& NGC 6426     & II         & Clement, 1991        &                                                                        \\
		M3             & NGC 5272     & I          & Smolec, 2017 \cite{smolec_2017}         &                                                                        \\
		& NGC 6362     & I          & Smolec, 2017         &                                                                        \\
		Omega Centauri & NGC 5139     & II         & Smolec, 2017         &                                                                        \\
		M62            & NGC 6266     & I          & Contreras, 2010 \cite{contreras_2010}      &                                                                        \\
		M15            & NGC 7078     & II         & Contreras, 2010      &                                                                        \\
		& NGC 6362     & I          & Contreras, 2010      &                                                                        \\
		M107           & NGC 6171     & I          & Contreras, 2010      &                                                                        \\
		M5             & NGC 5904     & I          & Contreras, 2010      &                                                                        \\
		& NGC 6229     & I          & Contreras, 2010      &                                                                        \\
		& NGC 6934     & I          & Contreras, 2010      &                                                                        \\
		M3             & NGC 5272     & I          & Contreras, 2010      &                                                                        \\
		M2             & NGC 7089     & II         & Contreras, 2010      &                                                                        \\
		& NGC 5286     & II         & Contreras, 2010      &                                                                        \\
		M55            & NGC 6809     & II         & Contreras, 2010      &                                                                        \\
		& NGC 4147     & I          & Contreras, 2010      &                                                                        \\
		& NGC 2298     & II         & Contreras, 2010      &                                                                        \\
		M68            & NGC 4590     & II         & Contreras, 2010      &                                                                        \\
		M92            & NGC 6341     & II         & Contreras, 2010      &                                                                        \\
		M9             & NGC 6333     & II         & Arellano Ferro, 2013 \cite{arellano_ferro_2013} &                                                                        \\
		M53            & NGC 5024     & II         & Arellano Ferro, 2013 &                                                                        \\
		M15            & NGC 7078     & II         & Arellano Ferro, 2013 &                                                                        \\
		M68            & NGC 4590     & II         & Arellano Ferro, 2013 &                                                                        \\
		& NGC 6388     & III        & Arellano Ferro, 2017 \cite{arellano_ferro_2017} &                                                                        \\
		& NGC 6441     & III        & Arellano Ferro, 2017 &                                                                        \\
		& NGC 1851     & I          & Arellano Ferro, 2017 &                                                                        \\
		& NGC 3201     & I          & Arellano Ferro, 2017 &                                                                        \\
		& NGC 4147     & I          & Arellano Ferro, 2017 &                                                                        \\
		M3             & NGC 5272     & I          & Arellano Ferro, 2017 &                                                                        \\
		M5             & NGC 5904     & I          & Arellano Ferro, 2017 &                                                                        \\
		M107           & NGC 6171     & I          & Arellano Ferro, 2017 &                                                                        \\
		& NGC 6229     & I          & Arellano Ferro, 2017 &                                                                        \\
		& NGC 6362     & I          & Arellano Ferro, 2017 &                                                                        \\
		& NGC 6366     & I          & Arellano Ferro, 2017 &                                                                        \\
		& NGC 6934     & I          & Arellano Ferro, 2017 &                                                                        \\
		M72            & NGC 6981     & I          & Arellano Ferro, 2017 &                                                                        \\
		& NGC 288      & II         & Arellano Ferro, 2017 &                                                                        \\
		M79            & NGC 1904     & II         & Arellano Ferro, 2017 &                                                                        \\
		M68            & NGC 4590     & II         & Arellano Ferro, 2017 &                                                                        \\
		M53            & NGC 5024     & II         & Arellano Ferro, 2017 &                                                                        \\
		& NGC 5053     & II         & Arellano Ferro, 2017 &                                                                        \\
		& NGC 5466     & II         & Arellano Ferro, 2017 &                                                                        \\
		M9             & NGC 6333     & II         & Arellano Ferro, 2017 &                                                                        \\
		M92            & NGC 6341     & II         & Arellano Ferro, 2017 &                                                                        \\
		M15            & NGC 7078     & II         & Arellano Ferro, 2017 &                                                                        \\
		M2             & NGC 7089     & II         & Arellano Ferro, 2017 &                                                                        \\
		M30            & NGC 7099     & II         & Arellano Ferro, 2017 &                                                                        \\
		& NGC 7492     & II         & Arellano Ferro, 2017 &                                                                        \\
		M3             & NGC 5272     & I          & Jeffery, 2011 \cite{jeffery_2011}        &                                                                        \\
		M15            & NGC 7078     & II         & Stetson, 2014 \cite{stetson_2014}        &                                                                        \\
		M68            & NGC 4590     & II         & Stetson, 2014        &                                                                        \\
		M22            & NGC 6656     & II         & Stetson, 2014        &                                                                        \\
		& NGC 3201     & I          & Stetson, 2014        &                                                                        \\
		& NGC 1851     & I          & Stetson, 2014        &                                                                        \\
		& NGC 4147     & I          & Stetson, 2014        &                                                                        \\
		M54            & NGC 6715     & I          & Stetson, 2014        & This and M4 could be Oo-Int depending on the classification diagnostic \\
		M4             & NGC 6121     & Oo-Neutral & Stetson, 2014        &                                                                        \\
		M5             & NGC 5904     & Oo-Neutral & Stetson, 2014        &                                                                        \\
		M9             & NGC 6333     & II         & Clement, 2001 \cite{clement_2001_a}        &                                                                        \\
		M68            & NGC 4590     & II         & Clement, 2001        &                                                                        \\
		M55            & NGC 6809     & II         & Clement, 2001        &                                                                        \\
		& NGC 6426     & II         & Clement, 2001        &                                                                        \\
		M2             & NGC 7089     & II         & Clement, 2001        &                                                                        \\
		M53            & NGC 5024     & II         & Clement, 2001        &                                                                        \\
		M92            & NGC 6341     & II         & Clement, 2001        &                                                                        \\
		M15            & NGC 7078     & II         & Clement, 2001        &                                                                        \\
		M3             & NGC 5272     & I          & Clement, 2001        &                                                                        \\
		M107           & NGC 6171     & I          & Clement, 2001        &                                                                        \\
		M5             & NGC 5904     & I          & Clement, 2001        &                                                                        \\
		& NGC 6229     & I          & Clement, 2001        &                                                                        \\
		M15            & NGC 7078     & II         & Jang, 2014 \cite{jang_2014}           &                                                                        \\
		M3             & NGC 5272     & I          & Jang, 2014           &                                                                        \\
		M68            & NGC 4590     & II         & Jang, 2014           &                                                                        \\
		& NGC 5053     & II         & Jang, 2014           &                                                                        \\
		& NGC 5466     & II         & Jang, 2014           &                                                                        \\
		& NGC 6441     & III        & Jang, 2014           &                                                                        \\
		& NGC 1851     & I          & Kunder, 2013 \cite{kunder_2013_d}         &                                                                        \\
		& NGC 362      & I          & Kunder, 2013         &                                                                        \\
		& NGC 3201     & I          & Kunder, 2013         &                                                                        \\
		M68            & NGC 4590     & II         & Kunder, 2013         &                                                                        \\
		& IC 4499      & I          & Kunder, 2013         &                                                                        \\
		Rup 106        & Ruprecht 106 & I          & Kunder, 2013         &                                                                        \\
		& NGC 5053     & II         & Kunder, 2013         &                                                                        \\
		& NGC 6934     & I          & Kunder, 2013         &                                                                        \\
		M72            & NGC 6981     & I          & Kunder, 2013         &                                                                        \\
		& NGC 7006     & I          & Kunder, 2013         &                                                                        \\
		M15            & NGC 7078     & II         & Kunder, 2013         &                                                                        \\
		Omega Centauri & NGC 5139     & II         & Kunder, 2013         &                                                                        \\
		M3             & NGC 5272     & I          & Kunder, 2013         &                                                                        \\
		& NGC 5466     & II         & Kunder, 2013         &                                                                        \\
		& NGC 6229     & I          & Kunder, 2013         &                                                                        \\
		& NGC 6426     & II         & Kunder, 2013         &                                                                        \\
		& NGC 6441     & III        & Kunder, 2013         &                                                                        \\
		& NGC 6388     & III        & Kunder, 2013         &                                                                        \\
		M62            & NGC 6266     & I          & Kunder, 2013         &                                                                        \\
		M9             & NGC 6333     & II         & Kunder, 2013         &                                                                        \\
		M92            & NGC 6341     & II         & Kunder, 2013         &                                                                        \\
		& NGC 6362     & I          & Kunder, 2013         &                                                                        \\
		M28            & NGC 6626     & I          & Kunder, 2013         &                                                                        \\
		M5             & NGC 5904     & I          & Kunder, 2013         &                                                                        \\
		& NGC 5986     & II         & Kunder, 2013         &                                                                        \\
		M4             & NGC 6121     & I          & Kunder, 2013         &                                                                        \\
		M107           & NGC 6171     & I          & Kunder, 2013         &                                                                        \\
		& NGC 6712     & I          & Kunder, 2013         &                                                                        \\
		& NGC 6723     & I          & Kunder, 2013         &                                                                        \\
		M75            & NGC 6864     & I          & Kunder, 2013         &                                                                        \\
		M53            & NGC 5024     & II         & Kunder, 2013         &                                                                        \\
		M2             & NGC 7089     & II         & Kunder, 2013         &                                                                        \\
		& NGC 2419     & II         & Kunder, 2013         &                                                                        \\
		& NGC 1851     & I          & Kunder, 2013         &                                                                        \\
		& NGC 5286     & II         & Kunder, 2013         &                                                                        \\
		& NGC 2808     & I          & Kunder, 2013         &                                                                        \\
		& NGC 4147     & I          & Kunder, 2013         &                                                                        \\
		M55            & NGC 6809     & I          & Kunder, 2013         &                                                                        \\
		M54            & NGC 6715     & I          & Kunder, 2013         &                                                                        \\
		M79            & NGC 1904     & II         & Kunder, 2013         &                                                                        \\
		M3             & NGC 5272     & I          & Pritzl, 2000 \cite{pritzl_2000_a}         &                                                                        \\
		M15            & NGC 7078     & II         & Pritzl, 2000         &                                                                        \\
		M3             & NGC 5272     & I          & Kinemuchi, 2006 \cite{kinemuchi_2006_a}      &                                                                        \\
		Omega Centauri & NGC 5139     & II         & Kinemuchi, 2006      &                                                                       
\end{longtable}

\newpage

\subsubsection{Globular Cluster Data Releases}
\label{sec:app_data_releases}

This table contains information on various data releases on globular cluster RR Lyrae data. This data was intended to be used to be used to evaluate the clustering of the LMC and SMC data in terms of its relationship to the Oosterhoff dichotomy.

\vspace{12pt}

This data was collected by looking for data releases on the VizieR Catalog and the SAO/NASA Astrophysics Data System (ADS) using search terms like “RR Lyrae”. The data releases were filtered down to data releases that fit the following criteria.

\begin{itemize}
	\item Included RR Lyrae data for a specific globular cluster.
	\item The globular cluster that the data release was for fit the criteria specified in section \ref{sec:app_globular_clusters_summary} \nameref{sec:app_globular_clusters_summary}.
\end{itemize}

\textbf{Link:} \url{https://github.com/ExcaliburZero/rr-lyrae-oosterhoff-clustering/blob/master/data/raw/gc_oosterhoff/Collected\%20Globular\%20Cluster\%20Information\%20-\%20Data\%20Releases.csv}

\paragraph{Columns}

\begin{outline}
	\1 Name
	\2 The non-NGC name of the globular cluster. (if any) (ex. M3, M5, M72)
	\1 NGC
	\2 The NGC designation of the globular cluster, or the globular cluster’s name if an NGC designation could not be found. (ex. NGC 5272, NGC 5904, NGC 6121)
	\1 Oost Type
	\2 The Oosterhoff type that the paper suggests that the globular cluster is. (I, II, III, Oo-Int, Oo-Neutral)
	\1 Bands
	\2 The photometric bands that the data release contains.
	\1 Paper
	\2 The paper associated with the data release.
	\1 Link
	\2 A link to the data release.
\end{outline}

\paragraph{Sources}

\begin{itemize}
	\item Alves, 2001 \cite{alves_2001}
	\item Arellano Ferro, 2004 \cite{arellano_ferro_2004}
	\item Arellano Ferro, 2006 \cite{arellano_ferro_2006}
	\item Arellano Ferro, 2008 \cite{arellano_ferro_2008}
	\item Arellano Ferro, 2010 \cite{arellano_ferro_2010}
	\item Arellano Ferro, 2011 \cite{arellano_ferro_2011}
	\item Arellano Ferro, 2012 \cite{arellano_ferro_2012}
	\item Arellano Ferro, 2013 \cite{arellano_ferro_2013}
	\item Arellano Ferro, 2014 \cite{arellano_ferro_2014}
	\item Arellano Ferro, 2015 \cite{arellano_ferro_2015}
	\item Arellano Ferro, 2016 \cite{arellano_ferro_2016}
	\item Braga, 2016 \cite{braga_2016}
	\item Brocato, 1994 \cite{brocato_1994}
	\item Cacciari, 2005 \cite{cacciari_2005}
	\item Clement, 1991 \cite{clement_1991_b}
	\item Clement, 1993 \cite{clement_1993}
	\item Clement, 1996 \cite{clement_1996}
	\item Clement, 1997 \cite{clement_1997}
	\item Clement, 1999 \cite{clement_1999}
	\item Clement, 2000 \cite{clement_2000}
	\item Contreras, 2010 \cite{contreras_2010}
	\item Corwin, 2008 \cite{corwin_2008}
	\item Dekany, 2009 \cite{dekany_2009}
	\item Di Criscienzo, 2011 \cite{di_criscienzo_2011_b}
	\item Fernandez-Trincado, 2015 \cite{fernandez-trincado_2015}
	\item Jurcsik, 2012 \cite{jurcsik_2012}
	\item Jurcsik, 2015 \cite{jurcsik_2015}
	\item Jurcsik, 2017 \cite{jurcsik_2017}
	\item Kains, 2015 \cite{kains_2015}
	\item Kaluzny, 1997 \cite{kaluzny_1997}
	\item Montiel, 2010 \cite{montiel_2010}
	\item Nemec, 1995 \cite{nemec_1995}
	\item Petersen 1994 \cite{petersen_1994}
	\item Siegel, 2015 \cite{siegel_2015}
	\item Silbermann, 1995 \cite{silbermann_1995}
	\item Sollima, 2006 \cite{sollima_2006}
	\item Stetson, 2014 \cite{stetson_2014}
	\item Szeidl, 2011 \cite{szeidl_2011}
	\item Vivas, 2017 \cite{vivas_2017}
	\item Sources for Oosterhoff classifications are listed in section~\ref{sec:app_oosterhoff_classifications}~\nameref{sec:app_oosterhoff_classifications}.
\end{itemize}

\newpage

\subsection{Reproducibility Directions}
All of the code used for the data analysis performed is available at the GitHub repository below.

\vspace{12pt}

\url{https://github.com/ExcaliburZero/rr-lyrae-oosterhoff-clustering}

\vspace{12pt}

The repository includes a Makefile, which can be used to download all of the data from the OGLE website, perform the data analysis, and generate the reports with the results.

\vspace{12pt}

The data analysis scripts were written on an Ubuntu Linux machine, so they should work on all Linux distributions and Mac. However, no guarantee as to whether they will work on a Windows machine.

\vspace{12pt}

The data analysis scripts depend on the following software, so to re-run the analysis you will need to install the following programs.

\begin{itemize}
	\item Pipenv	-	\url{https://pipenv.readthedocs.io/en/latest/}
	\item Pyenv		-	\url{https://github.com/pyenv/pyenv}
\end{itemize}

The data analysis scripts are written in a combination of Python and R. Most of the scripts that do that actual analysis are written in Python, and some of the plotting scripts are written in R.

\vspace{12pt}

The dependencies for the Python scripts are all handled by Pipenv, but the following R libraries are needed for the R scripts.

\begin{itemize}
	\item ggplot2
	\item repr
	\item ggrepel
	\item plyr
\end{itemize}

These libraries can be installed by running the following.

\begin{lstlisting}
$ R
> install.packages("ggplot2")
> install.packages("repr")
> install.packages("ggrepel")
> install.packages("plyr")
\end{lstlisting}


Once you have these dependencies installed, you can download and run the analysis scripts by running the following terminal commands.

\vspace{12pt}

\begin{lstlisting}[language=bash]
$ git clone https://github.com/ExcaliburZero/rr-lyrae-oosterhoff-clustering
$ pipenv install
$ pipenv shell
$ make
\end{lstlisting}

\newpage

\includepdf[
%% Include all pages of the PDF
pages=-,
%% make this page have the usual page style
%% (you can change it to plain etc). By default pdfpages
%% sets the pagecommand to \pagestyle{empty}
%pagecommand={\pagestyle{headings}},  
pagecommand={},  
%% Add a "section" entry to the ToC with the heading
%% "Quilling Shapes" and the label "sec:shapes"
addtotoc={1,subsection,1,Data Analysis Jupyter Notebook,sec:data-analysis}]
%% The pdf file itself
{RRab_OGLE_IV_Clustering.pdf}

\end{document}
