\documentclass[]{article}

%opening
\title{}
\author{Christopher Wells}

\begin{document}

\maketitle

\newpage

\begin{abstract}
	
	RR Lyrae are a type of periodic variable star that are often examined due to their utility as distance indicators in space. An effect known as the Oosterhoff Dichotomy has been seen in populations of RR Lyrae stars in globular clusters in the Milky Way Galaxy. This phenomenon is thought to have some importance in learning more about the formation of the Milky Way Galaxy.

\end{abstract}

\newpage

\tableofcontents

\newpage

\section{Advice to Future Honors Thesis Students}

\newpage

\section{Acknowledgements}

\newpage

\section{Author's Reflections}

\newpage

\section{RR Lyrae}

RR Lyrae are a type of radially pulsating, periodic variable star that are numerous in Local Group galaxies and examined due to their utility as distance indicators and indicators of early stellar composition. (Szczygiel et al. 2009, Kinemuchi et al. 2006 (Analysis of ...)) RR Lyrae are advanced age stars (> 10 Gyrs) that undergo radial pulsations, whereby they expand and contract radially over time. (Szczygiel et al. 2009, ?) These pulsations that RR Lyrae undergo cause the brightness of the star to change over time, with it becoming brighter as it expands and dimmer as it contracts. (?) This change in brightness over time classifies RR Lyrae as variable stars, however unlike some variable stars, RR Lyrae change in brightness at a regular interval, or period, which makes them periodic variable stars. (?) RR Lyrae are numerous in the Milky Way and other Local Group galaxies and are additionally easy to identify. (Soszyński, 2016, Kinemuchi et al. 2006 (Analysis of ...)) RR Lyrae are often researched due to their use as “standard candles”, or distance indicators.\cite{2016AcA....66..131S} (Soszyński, 2016) The intensity of the luminosity change caused by the periodic pulsations is correlated with the period of the pulsations, thus the period can be used to calculate the luminosity of the star which can be compared with the observed brightness of the star to determine its distance. (?)

...

\newpage

\section{Oosterhoff Dichotomy}

\newpage

\section{Clustering}

\newpage

\section{Analysis}

\newpage

\section{References}

\bibliography{references} 
\bibliographystyle{ieeetr4}

\newpage

\section{Appendices}

\end{document}
